%!TEX root = ../dissertation.tex

\begin{otherlanguage}{portuguese}
\begin{abstract}
A Unidade Curricular de Arquitecturas de Software no Instituto Superior T\'{e}cnico ensina aos alunos os conceitos mais importantes sobre o desenho e arquitectura de sistemas de software, e ajuda estes alunos a aplicar estes mesmos conceitos a sistemas de software reais e complexos. Organizar os conhecimentos e aplicar a teoria \`{a} pr\'{a}tica n\~{a}o \'{e} uma tarefa f\'{a}cil, e muitas vezes os alunos precisam de colocar quest\~{o}es e discutir problemas. O estado da arte sobre software social e colaborativo e estrutura\c{c}\~{a}o de conhecimentos foi analisado e uma plataforma social foi desenvolvida para ajudar a resolver estes problemas

% Keywords
\begin{flushleft}
\keywords{Software Social, Estrutura\c{c}\~{a}o de Conhecimentos, Plataforma Colaborativa, Sistemas de Reputa\c{c}\~{a}o, Sistemas de tags, Ontologia, Taxonomia}

\end{flushleft}

\end{abstract}
\end{otherlanguage}
