\chapter{Domain Model}
\label{chapter:domainModel}
To understand the architectural and implementation decisions taken during the solution development, it is necessary to understand the Software Architectures Domain Model. Section \ref{section:SAConcepts} lists and describes the concepts talked in the Software Architectures classes, Section \ref{section:domainModel} presents an abstract domain model where it is possible to see how these concepts relate and Section \ref{section:templates} gives an idea on how the information from the domain model can be represented in templates.

\section{Concepts}
\label{section:SAConcepts}
Before describing the domain model of the developed solution, it is necessary to provide a small introduction to the most important concepts from the Software Architectures course.
 
\subsection{Scenarios}
\label{subsection:SAConceptsScenarios}
A scenario is used to capture and express the quality requirements of a system. The considered qualities are:
\begin{itemize}
\item Availability - concerning the uptime of a system;
\item Interoperability - how easily can the system interoperate with other system;
\item Modifiability - the cost of changing the system;
\item Performance - how fast the system responds to events;
\item Security - resistance to unauthorized usage whilst providing service to legitimate users;
\item Testability - how easy it is to check system faults through testing;
\item Usability - how easy it is for a user to accomplish a task, and the kind of support offered by the system;
\end{itemize}

A scenario consists of six parts \cite{bass2003software}:
\begin{itemize}
\item \textbf{Source of Stimulus:} Some entity (a human, a computer or any other actuator) that generates the stimulus;
\item \textbf{Stimulus:} A condition that arrives at the system;
\item \textbf{Environment:} The system condition when the stimulus occurs;
\item \textbf{Artifact:} Part of the system that was stimulated;
\item \textbf{Response:} Activity undertaken after the arrival of the stimulus;
\item \textbf{Response Measure:} when the response occurs, it should be measurable in some way, so the requirement can be tested;
\end{itemize}

Each Scenario uses a set of Tactics, which are design decisions used to achieve the quality requirements expressed in them. Each quality requirement has a set of commonly used tactics. For example, to assure the Security of a system, tactics such as detecting service denial or message delays (to detect attacks), or revoking access to the system (to react to an attack) are used.

\subsection{Views}
\label{subsection:domainModelViews}
A view is a representation of a set of system elements and the relationships associated with them \cite{clements2003documenting}. This set of elements and relationships is constrained by viewtypes.

A Viewtype defines the element types and relationship types used to describe the architecture of a software system from a particular perspective. Viewtypes refine into styles.

An architectural style is a specialization of element and relation types, together with a set of constraints on how they can be used.

Views can fall into three viewtype categories:
\begin{itemize}
\item \textbf{Module Viewtype:} document the system principal units of implementation. 

The elements of this viewtype are the \textit{Modules}, which is are implementation units. 

Relationships between modules can be of type \textit{``Is-part-of''}, which defines a part-whole relationship, \textit{``Depends on''}, which defines dependency relations, and \textit{``Is-a''}, which defines a generalization/specialization relationship.

\item \textbf{Component \& Connector Viewtype:} document the system units of execution. 

The elements are the Components, which are the principal processing units and data stores, and the Connectors, which are pathways of interaction between components. 

The relationships can be of type \textit{``Attachment''}, which associate components to connectors, and \textit{``Interface''} Delegation, which associates component ports to other ports from an ``internal architecture''- and similarly for the connector.

\item \textbf{Allocation Viewtype:} document the relationships between a system's software and its development and execution environments. 

The elements are the \textit{Software Element} (elements from the Module and Component \& Connector viewtypes) and the \textit{Environmental Element}. 

The relationships are of type \textit{``Allocated-to''}, which means that a software element is allocated to an environmental element. 
\end{itemize}

\section{Model}
\label{section:domainModel}
Figure \ref{figure:abstractDomainModel} shows the concepts and relations described in Section \ref{section:SAConcepts} in a Domain Model Diagram.
 
\begin{figure}[h]
\centering
\renewcommand {\umltextcolor}{black}
\renewcommand {\umlfillcolor}{none}
\renewcommand {\umldrawcolor}{black}

\begin{tikzpicture}
\tikzstyle{every node}=[font=\footnotesize]
\begin{class}[text width=2cm]{Template Concept}{10,-4.5}
\end{class}

\begin{class}[text width=3cm ]{Source Of Stimulus}{0,0}

\end{class}
\begin{class}[text width=2cm ]{Stimulus}{0,-1.5}
%\inherit{ScenarioElement}
\end{class}
\begin{class}[text width=2cm ]{Artifact}{0,-2.5}
%\inherit{ScenarioElement}
\end{class}
\begin{class}[text width=2.5cm ]{Environment}{0,-3.5}
%\inherit{ScenarioElement}
\end{class}
\begin{class}[text width=2cm ]{Response}{0,-4.5}
%\inherit{ScenarioElement}
\end{class}
\begin{class}[text width=3.3cm ]{Response Measure}{0,-5.5}
\end{class}
\begin{class}[text width=2cm ]{Scenario}{4.5,-2.5}
\inherit{Template Concept}
\end{class}
\begin{class}[text width=3cm ]{Quality Requirement}{0,3}
\end{class}
\begin{class}[text width=2cm ]{Tactic}{5,2}
\end{class}

\begin{class}[text width=2cm ]{View}{4,-7.5}
\inherit{Template Concept}
\end{class}
\begin{class}[text width=2cm ]{Viewtype}{0,-7}
\end{class}
\begin{class}[text width=2cm ]{Style}{-1,-8.5}
\end{class}
\begin{class}[text width=2cm ]{Relation}{-2,-10}
\end{class}

\begin{class}[text width=2cm ]{Element}{5,-10}
\inherit{Template Concept}
\end{class}

\begin{class}[text width=1.5cm]{Is Part Of}{-4,-11.5}
\inherit{Relation}
\end{class}

\draw [umlcd style] (-2.5,-11.90) node {[...]};

\begin{class}[text width=2cm]{Attachment}{-1,-11.5}
\inherit{Relation}
\end{class}

\begin{class}[text width=2cm]{Allocated To}{1.5,-11.5}
\inherit{Relation}
\end{class}

\begin{class}[text width=2cm]{Module}{4,-11.5}
\inherit{Element}
\end{class}

\begin{class}[text width=2cm]{Component}{6.5,-11.5}
\inherit{Element}
\end{class}

\begin{class}[text width=2.5cm]{Environmental Element}{9.5,-11.5}
\inherit{Element}
\end{class}

\begin{class}[text width=2.5cm]{Connector}{10,-10}
\inherit{Element}
\end{class}

\association{Source Of Stimulus}{}{0..1}{Scenario}{}{1}
\association{Stimulus}{}{0..1}{Scenario}{}{1}
\association{Artifact}{}{0..1}{Scenario}{}{1}
\association{Environment}{}{0..1}{Scenario}{}{1}
\association{Response}{}{0..1}{Scenario}{}{1}
\association{Response Measure}{}{0..1}{Scenario}{}{1}
\association{Quality Requirement}{}{1}{Scenario}{}{*}
\association{Quality Requirement}{}{1}{Tactic}{}{*}
\association{Scenario}{}{*}{Tactic}{}{*}
\association{View}{}{*}{Viewtype}{}{1}
\association{Viewtype}{}{1}{Style}{}{*}
\association{Style}{}{*}{Element}{}{*}
\association{Style}{}{*}{Relation}{}{*}
\association{View}{}{*}{Element}{}{*}
\association{Element}{}{*}{Relation}{}{*}



\end{tikzpicture}
\caption{Domain Model showing the Software Architectures concepts and how they are related}
\label{figure:abstractDomainModel}
\end{figure}

The scenario elements are represented as the model entities 'Source of Stimulus', 'Stimulus', Artifact', 'Environment', 'Response' and 'Response Measure' respectively, and are associated to the Scenario entity. A Scenario can only have at most one instance of each element, hence the ``0..1'' cardinality. Similarly, each scenario element can only be associated with a single scenario, and therefore there is a ``1'' cardinality in the diagram. Each Scenario captures a single Quality Requirement and has a set of Tactics. As mentioned, each quality requirement has a set of commonly used tactics for its achievement.

A View is associated with a Viewtype, which refines into a style. The View contains a set of elements, which are related to each other via specific relations.

The Scenario, View and Element are considered the main concepts of the domain, and they have a dedicated template in the developed solution, as it will be described in the next section. These templates will contain information about the concept and how it is related with other concepts in the domain model.

\section{Templates}
\label{section:templates}
The structured representation of the Software Architectures concepts described in Sections \ref{section:SAConcepts} and \ref{section:domainModel} is represented in the developed solution by using specific templates for these concepts.

Although we can elicit a wide set of concepts, it does not make sense to have a template for each and every one of them. It is easier to see the relations between concepts if they are in the same template. 

This is the case for the Scenarios. It makes sense to see all the elements of a Scenario together, so it is possible to see, for example, who/what generated the stimulus and what part of the system was stimulated. Therefore, the Scenario is considered one of the main concepts, and it has its own dedicated template, in which are present the quality attribute, the elements and the tactics. 

Figure \ref{figure:scenarioTemplate} shows a Schema for the Scenario template. All the Scenario elements, the quality requirement and the tactics are present in the template, making it possible to see, as mentioned before, how all these concepts are related.

\begin{figure}[h]
\centering
\lstinputlisting[language=HTML, style=customhtml]{scenarioSchema.html}
\caption{Schema for the Scenario template}
\label{figure:scenarioTemplate}
\end{figure}

The Views and their elements are also another case of main concepts. It makes sense to aggregate all the system elements and their relationships from a view in a separate template, representing the part of the system being described in that view. However, since a single element can be present in more than one view, it also makes sense to have a specific template for the elements, so their individual properties can be seen.

Figure \ref{figure:moduleTemplate} shows a Schema for the Module template, which is the element from the Module Viewtype. Besides the element details, its relationships with other elements (other Modules, in this specific example) are present in the template.

\begin{figure}[h]
\centering
\lstinputlisting[language=HTML, style=customhtml]{moduleSchema.html}
\caption{Schema for the Module template}
\label{figure:moduleTemplate}
\end{figure}

Figure \ref{figure:viewTemplate} shows the schema for the Module Viewtype Views template. Although the template allows inserting a description for the view, its most important feature is the inclusion of other templates in it. As mentioned, a view aggregates a set of system elements and the relationships between them. As there is already a template for the element, the view will include a set of element templates - a template per element, with its respective information.

\begin{figure}[h]
\centering
\lstinputlisting[language=HTML, style=customhtml]{viewSchema.html}
\caption{Schema for a Module Viewtype View template}
\label{figure:viewTemplate}
\end{figure}

