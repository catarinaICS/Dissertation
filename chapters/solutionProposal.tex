%TEX root = ../dissertation.tex

\chapter{Solution}
\label{chapter:solution}
\label{solarch}
To solve the problems elicited throughout this document, it was developed a Web Application. This kind of application facilitates collaboration between its users, and therefore was the more adequate choice for the system to develop. The developed application features:
\begin{itemize}
\item An authentication system, where users can login into the system. This authentication system provides an unique identity for a user inside the application, and also provides distinction between types of users, as there are users of type STUDENT and users of type TEACHER, with different permissions.
\item Document Management, only available for teachers, where software description articles can be added or removed;
\item Document parsing into a view;
\item Creation of annotations in the document text. As the name says, an annotation is a part of the text that is marked and highlighted. The AnnotatorJS (REFER) library provides tools not only to mark up parts of text, but also to associate tags and user text to that selected text. 
\item Templates for a set of Software Architectures concepts. The parts of marked text described before can be associated to these templates, thus making it easier to co-relate the theoretical concepts with the practical applications. The concepts used in these templates are represented as entities of the domain model.

\end{itemize}

The next chapter will describe the developed system architecture.