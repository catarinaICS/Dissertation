%TEX root = ../dissertation.tex

\chapter{Solution}
\label{chapter:solution}
To solve the problems elicited throughout this document, it was developed a Web Application to be used by students and teachers, both in class and home environments. This kind of application facilitates collaboration between its users, and therefore was the more adequate choice for the system to develop. The developed application features:
\begin{itemize}
\item An authentication system, where users can login into the system. This authentication system provides an unique identity for a user inside the application, and also provides distinction between types of users, as there are users of type STUDENT and users of type TEACHER, with different permissions.
\item Document Management, only available for teachers, where software description articles can be added or removed;
\item Document parsing into a view;
\item Creation of annotations in the document text. As the name says, an annotation is a part of the text that is marked and highlighted. The AnnotatorJS \footnote{http://annotatorjs.org/} library provides tools not only to mark up parts of text, but also to associate tags and user text to that selected text. 
\item Templates for a set of Software Architectures concepts. The parts of marked text described before can be associated to these templates, thus making it easier to co-relate the theoretical concepts with the practical applications. The concepts used in these templates are represented as entities of the domain model.

\end{itemize}

The next sections will give an overview of the domain and architecture of the developed system.

\section{Domain Model}
\label{chapter:domainModel}
To understand the architectural and implementation decisions taken during the solution development, it is necessary to understand the Software Architectures Domain Model. Section \ref{section:SAConcepts} lists and describes the concepts talked in the Software Architectures classes, Section \ref{section:domainModel} presents an abstract domain model where it is possible to see how these concepts relate and Section \ref{section:templates} gives an idea on how the information from the domain model can be represented in templates.

\subsection{Concepts}
\label{section:SAConcepts}
Before describing the domain model of the developed solution, it is necessary to provide a small introduction to the most important concepts from the Software Architectures course.
 
\subsubsection{Scenarios}
\label{subsection:SAConceptsScenarios}
A scenario is used to capture and express the quality requirements of a system. The considered qualities are:
\begin{itemize}
\item Availability - concerning the uptime of a system;
\item Interoperability - how easily can the system interoperate with other system;
\item Modifiability - the cost of changing the system;
\item Performance - how fast the system responds to events;
\item Security - resistance to unauthorized usage whilst providing service to legitimate users;
\item Testability - how easy it is to check system faults through testing;
\item Usability - how easy it is for a user to accomplish a task, and the kind of support offered by the system;
\end{itemize}

A scenario consists of six parts \cite{bass2003software}:
\begin{itemize}
\item \textbf{Source of Stimulus:} Some entity (a human, a computer or any other actuator) that generates the stimulus;
\item \textbf{Stimulus:} A condition that arrives at the system;
\item \textbf{Environment:} The system condition when the stimulus occurs;
\item \textbf{Artifact:} Part of the system that was stimulated;
\item \textbf{Response:} Activity undertaken after the arrival of the stimulus;
\item \textbf{Response Measure:} when the response occurs, it should be measurable in some way, so the requirement can be tested;
\end{itemize}

Each Scenario uses a set of Tactics, which are design decisions used to achieve the quality requirements expressed in them. Each quality requirement has a set of commonly used tactics. For example, to assure the Security of a system, tactics such as detecting service denial or message delays (to detect attacks), or revoking access to the system (to react to an attack) are used.

\subsubsection{Views}
\label{subsection:domainModelViews}
A view is a representation of a set of system elements and the relationships associated with them \cite{clements2003documenting}. This set of elements and relationships is constrained by viewtypes.

A Viewtype defines the element types and relationship types used to describe the architecture of a software system from a particular perspective. Viewtypes refine into styles.

An architectural style is a specialization of element and relation types, together with a set of constraints on how they can be used.

Views can fall into three viewtype categories:
\begin{itemize}
\item \textbf{Module Viewtype:} document the system principal units of implementation. 

The elements of this viewtype are the \textit{Modules}, which is are implementation units. 

Relationships between modules can be of type \textit{``Is-part-of''}, which defines a part-whole relationship, \textit{``Depends on''}, which defines dependency relations, and \textit{``Is-a''}, which defines a generalization/specialization relationship.

\item \textbf{Component \& Connector Viewtype:} document the system units of execution. 

The elements are the Components, which are the principal processing units and data stores, and the Connectors, which are pathways of interaction between components. 

The relationships can be of type \textit{``Attachment''}, which associate components to connectors, and \textit{``Interface''} Delegation, which associates component ports to other ports from an ``internal architecture''- and similarly for the connector.

\item \textbf{Allocation Viewtype:} document the relationships between a system's software and its development and execution environments. 

The elements are the \textit{Software Element} (elements from the Module and Component \& Connector viewtypes) and the \textit{Environmental Element}. 

The relationships are of type \textit{``Allocated-to''}, which means that a software element is allocated to an environmental element. 
\end{itemize}

\subsection{Model}
\label{section:domainModel}
Figure \ref{figure:abstractDomainModel} shows the concepts and relations described in Section \ref{section:SAConcepts} in a Domain Model Diagram.
 
\begin{figure}[h]
\centering
\renewcommand {\umltextcolor}{black}
\renewcommand {\umlfillcolor}{none}
\renewcommand {\umldrawcolor}{black}

\begin{tikzpicture}
\tikzstyle{every node}=[font=\footnotesize]
\begin{class}[text width=2cm]{Template Concept}{10,-4.5}
\end{class}

\begin{class}[text width=3cm ]{Source Of Stimulus}{0,0}

\end{class}
\begin{class}[text width=2cm ]{Stimulus}{0,-1.5}
%\inherit{ScenarioElement}
\end{class}
\begin{class}[text width=2cm ]{Artifact}{0,-2.5}
%\inherit{ScenarioElement}
\end{class}
\begin{class}[text width=2.5cm ]{Environment}{0,-3.5}
%\inherit{ScenarioElement}
\end{class}
\begin{class}[text width=2cm ]{Response}{0,-4.5}
%\inherit{ScenarioElement}
\end{class}
\begin{class}[text width=3.3cm ]{Response Measure}{0,-5.5}
\end{class}
\begin{class}[text width=2cm ]{Scenario}{4.5,-2.5}
\inherit{Template Concept}
\end{class}
\begin{class}[text width=3cm ]{Quality Requirement}{0,3}
\end{class}
\begin{class}[text width=2cm ]{Tactic}{5,2}
\end{class}

\begin{class}[text width=2cm ]{View}{4,-7.5}
\inherit{Template Concept}
\end{class}
\begin{class}[text width=2cm ]{Viewtype}{0,-7}
\end{class}
\begin{class}[text width=2cm ]{Style}{-1,-8.5}
\end{class}
\begin{class}[text width=2cm ]{Relation}{-2,-10}
\end{class}

\begin{class}[text width=2cm ]{Element}{5,-10}
\inherit{Template Concept}
\end{class}

\begin{class}[text width=1.5cm]{Is Part Of}{-4,-11.5}
\inherit{Relation}
\end{class}

\draw [umlcd style] (-2.5,-11.90) node {[...]};

\begin{class}[text width=2cm]{Attachment}{-1,-11.5}
\inherit{Relation}
\end{class}

\begin{class}[text width=2cm]{Allocated To}{1.5,-11.5}
\inherit{Relation}
\end{class}

\begin{class}[text width=2cm]{Module}{4,-11.5}
\inherit{Element}
\end{class}

\begin{class}[text width=2cm]{Component}{6.5,-11.5}
\inherit{Element}
\end{class}

\begin{class}[text width=2.5cm]{Environmental Element}{9.5,-11.5}
\inherit{Element}
\end{class}

\begin{class}[text width=2.5cm]{Connector}{10,-10}
\inherit{Element}
\end{class}

\association{Source Of Stimulus}{}{0..1}{Scenario}{}{1}
\association{Stimulus}{}{0..1}{Scenario}{}{1}
\association{Artifact}{}{0..1}{Scenario}{}{1}
\association{Environment}{}{0..1}{Scenario}{}{1}
\association{Response}{}{0..1}{Scenario}{}{1}
\association{Response Measure}{}{0..1}{Scenario}{}{1}
\association{Quality Requirement}{}{1}{Scenario}{}{*}
\association{Quality Requirement}{}{1}{Tactic}{}{*}
\association{Scenario}{}{*}{Tactic}{}{*}
\association{View}{}{*}{Viewtype}{}{1}
\association{Viewtype}{}{1}{Style}{}{*}
\association{Style}{}{*}{Element}{}{*}
\association{Style}{}{*}{Relation}{}{*}
\association{View}{}{*}{Element}{}{*}
\association{Element}{}{*}{Relation}{}{*}



\end{tikzpicture}
\caption{Domain Model showing the Software Architectures concepts and how they are related}
\label{figure:abstractDomainModel}
\end{figure}

The scenario elements are represented as the model entities 'Source of Stimulus', 'Stimulus', Artifact', 'Environment', 'Response' and 'Response Measure' respectively, and are associated to the Scenario entity. A Scenario can only have at most one instance of each element, hence the ``0..1'' cardinality. Similarly, each scenario element can only be associated with a single scenario, and therefore there is a ``1'' cardinality in the diagram. Each Scenario captures a single Quality Requirement and has a set of Tactics. As mentioned, each quality requirement has a set of commonly used tactics for its achievement.

A View is associated with a Viewtype, which refines into a style. The View contains a set of elements, which are related to each other via specific relations.

The Scenario, View and Element are considered the main concepts of the domain, and they have a dedicated template in the developed solution, as it will be described in the next section. These templates will contain information about the concept and how it is related with other concepts in the domain model.

\subsection{Templates}
\label{section:templates}
The structured representation of the Software Architectures concepts described in \ref{section:SAConcepts} and \ref{section:domainModel} is represented in the developed solution by using specific templates for these concepts.

Although we can elicit a wide set of concepts, it does not make sense to have a template for each and every one of them. It is easier to see the relations between concepts if they are in the same template. 

This is the case for the Scenarios. It makes sense to see all the elements of a Scenario together, so it is possible to see, for example, who/what generated the stimulus and what part of the system was stimulated. Therefore, the Scenario is considered one of the main concepts, and it has its own dedicated template, in which are present the quality attribute, the elements and the tactics. 

Figure \ref{figure:scenarioTemplate} shows a Schema for the Scenario template. All the Scenario elements, the quality requirement and the tactics are present in the template, making it possible to see, as mentioned before, how all these concepts are related.

\begin{figure}[h]
\centering
\lstinputlisting[language=HTML, style=customhtml]{scenarioSchema.html}
\caption{Schema for the Scenario template}
\label{figure:scenarioTemplate}
\end{figure}

The Views and their elements are also another case of main concepts. It makes sense to aggregate all the system elements and their relationships from a view in a separate template, representing the part of the system being described in that view. However, since a single element can be present in more than one view, it also makes sense to have a specific template for the elements, so their individual properties can be seen.

Figure \ref{figure:moduleTemplate} shows a Schema for the Module template, which is the element from the Module Viewtype. Besides the element details, its relationships with other elements (other Modules, in this specific example) are present in the template.

\begin{figure}[h]
\centering
\lstinputlisting[language=HTML, style=customhtml]{moduleSchema.html}
\caption{Schema for the Module template}
\label{figure:moduleTemplate}
\end{figure}

Figure \ref{figure:viewTemplate} shows the schema for the Module Viewtype Views template. Although the template allows inserting a description for the view, its most important feature is the inclusion of other templates in it. As mentioned, a view aggregates a set of system elements and the relationships between them. As there is already a template for the element, the view will include a set of element templates - a template per element, with its respective information.

\begin{figure}[h]
\centering
\lstinputlisting[language=HTML, style=customhtml]{viewSchema.html}
\caption{Schema for a Module Viewtype View template}
\label{figure:viewTemplate}
\end{figure}

%!TEX root = ../dissertation.tex

\section{Architecture Analysis}
\label{chapter:architecture}
Section \ref{chapter:domainModel} introduced the domain model of the Software Architectures concepts present in the developed solution, and the templates in which they are included.

In this section, it is shown how these templates are filled with information extracted from a software description article.

The next sections will present two entities from the system: the Document and the Annotation. These entities have a major role in the identification of Software Architectures concepts from the domain model, and the enrichment of the templates for those concepts.

\subsection{Document}
\label{section:document}

The Document entity in this system corresponds to an article which describes a software system and is read and analyzed in the practical classes of the Software Architectures course. This entity saves the article text and its source, so it can be parsed locally in the application

Figure \ref{figure:documentEntity} shows the Document entity in the system. The ``title'' attribute corresponds to the title of the article, the ``url'' saves the article's original URL, and the ``content'' saves the article's contents.

\begin{figure}[h]
\centering
\renewcommand {\umltextcolor}{black}
\renewcommand {\umlfillcolor}{none}
\renewcommand {\umldrawcolor}{black}

\begin{tikzpicture}
\begin{class}[text width=3cm ]{Document}{0,0}
	\attribute { title : String }
	\attribute { url : String }
	\attribute { content : String }
\end{class}
\end{tikzpicture}
\caption{The Document Entity}
\label{figure:documentEntity}
\end{figure}

As the main purpose of this application is to create a structured representation of the document using the templates described in \ref{section:templates}, the instances of the main concepts, from which the templates will be generated, must be associated to the Document, as shown in Figure \ref{figure:documentConcept}. This way, generating a structured representation of the document is as easy as extracting the main concepts that are connected to it and including their respective templates in the structured representation. A concept describes a part of one and only one document, hence the '1' cardinality.  

\begin{figure}[h]
\centering
\renewcommand {\umltextcolor}{black}
\renewcommand {\umlfillcolor}{none}
\renewcommand {\umldrawcolor}{black}

\begin{tikzpicture}
\tikzstyle{every node}=[font=\footnotesize]
\begin{class}[text width=2cm ]{Template Concept}{5,0}
	\attribute{}
\end{class}

\begin{class}[text width=2cm ]{Document}{0,0}
	\attribute { title : String }
	\attribute { url : String }
	\attribute { content : String }
\end{class}

\association{Document}{}{1}{Template Concept}{}{*}
\end{tikzpicture}
\caption{Association between Document and Concept entities}
\label{figure:documentConcept}
\end{figure}

\subsection{Annotation}
\label{section:annotation}

An annotation consists of a portion of text selected from the article, enriched with a tag and other information. This selected text can correspond partially or totally to the specification of a Software Architecture concept. For example, the description of the Source of Stimulus of a Scenario may be found in a single paragraph of the article, or in a set of single sentences scattered along the whole article. 

Figure \ref{figure:annotationEntity} shows the Annotation entity in the system. The ``annotation'' field saves a JSON representation of the annotation data such as the quote from the text and the given tag. The ``tag'' field stores the tag given to the annotation. This tag corresponds to a Software Architectures concept, such as ``Stimulus'' or ``Module''.

\begin{figure}[h]
\centering
\renewcommand {\umltextcolor}{black}
\renewcommand {\umlfillcolor}{none}
\renewcommand {\umldrawcolor}{black}

\begin{tikzpicture}
\begin{class}[text width=4cm ]{Annotation}{0,0}
	\attribute { annotation : String }
	\attribute { tag : String }
\end{class}
\end{tikzpicture}
\caption{The Annotation Entity}
\label{figure:annotationEntity}
\end{figure}

Annotations are added to the system by selecting a portion of text in the parsed document and setting it as an annotation. Therefore, an annotation belongs to one, and only one document. In Figure \ref{figure:documentAnnotationConcepts} it is possible to see the association between the Annotation and Document entities, which reflects the mentioned constraints.

\begin{figure}[h]
\centering
\renewcommand {\umltextcolor}{black}
\renewcommand {\umlfillcolor}{none}
\renewcommand {\umldrawcolor}{black}

\begin{tikzpicture}
\tikzstyle{every node}=[font=\footnotesize]

\begin{class}[text width=2cm ]{Document}{0,0}
	\attribute { title : String }
	\attribute { url : String }
	\attribute { content : String }
\end{class}

\begin{class}[text width=3cm ]{Annotation}{-1,-3}
	\attribute { annotation : String }
	\attribute { tag : String }
\end{class}

\begin{class}[text width=2cm ]{Template Concept}{4,-0.5}
	\attribute{}
\end{class}

\begin{class}[text width=2cm ]{Scenario}{4,-3}
	\inherit{Template Concept}
	\attribute{}
\end{class}

\begin{class}[text width=3cm ]{Source Of Stimulus}{5,-5}
	\attribute{}
\end{class}

\association{Document}{}{1}{Template Concept}{}{*}
\association{Document}{}{1}{Annotation}{}{*}
\association{Template Concept}{}{0..1}{Annotation}{}{*}
\association{Scenario}{}{1}{Source Of Stimulus}{}{*}
\association{Source Of Stimulus}{}{0..1}{Annotation}{}{*}

\end{tikzpicture}
\caption{Associations between Document, Annotation and the domain model concepts}
\label{figure:documentAnnotationConcepts}
\end{figure}

Annotations are used to enrich the templates described in Section \ref{section:templates}. Associating the annotations with the respective instances of the concepts allows for that enrichment. All the concepts, not only the main ones, can have associated annotations.

As mentioned in the previous section, an annotation can partially or fully describe a concept, therefore a concept can be associated with multiple annotations. 

Figure \ref{figure:documentAnnotationConcepts} shows how the Annotation is associated with the domain concepts. There is an association not only with the main concepts of the domain model, but also all the other concepts represented in \ref{figure:abstractDomainModel}. The figure exemplifies these associations with the Scenario, which is a main concept, and the Source of Stimulus, but all the other concepts and main concepts - all the Scenario Elements, Modules, Components, etc. - have similar associations.  

\subsection{Interface Flow}
\label{section:interfaceFlow}

The associations between the entities described in the previous sections are created after a set of interactions with the system. These interactions are:

\begin{enumerate}
\item After navigation to the document page, a portion of text is selected. The AnnotatorJS interface is prompted, and the user selects a tag to describe the selected text;

\item Upon submission, an instance of ``Annotation'' is created and associated with the instance of the document;

\item By mouse hovering over the highlighted text, another AnnotatorJS interface is prompted, and the user can associate this annotation to a domain entity by clicking ``Add this annotation to the structured representation'';

\item Upon clicking the link, a modal window is shown and, according to the tag given to the annotation, the user can select either an existent Scenario, View or Module to add this annotation to, or create a new instance. 

Initially there are no created instances, and upon creation the instance is associated with the document. When a new Scenario is created, their elements are instantiated as well;

\item After selection, the annotation is associated with the selected element. In the case of Scenarios, if the annotation describes any of the scenario elements, it will be associated with the instance of the corresponding element. 

For example, an annotation tagged as ``Stimulus'' will be associated with the instance of the Stimulus entity that is associated with the selected Scenario.

\item The user is then redirected to the template of the selected element. Inside the template, the user can delete the annotation, move it to other element, delete the element itself and add text descriptions to the template. 

In the Scenario template the user can also add Tactics, in the Module template the user can add relations with other Modules, and in the View template the user can add elements to the view.
\end{enumerate}

Figures \ref{figure:scenarioTemplate2}, \ref{figure:moduleTemplate2} and \ref{figure:viewTemplate2} show the Schemas presented in \ref{section:templates}, now including the annotations and the user text.

\begin{figure}[h]
\centering
\lstinputlisting[language=HTML, style=customhtml]{scenarioSchema2.html}
\caption{Scenario Schema enriched with annotations and user text}
\label{figure:scenarioTemplate2}
\end{figure}

\begin{figure}[h]
\centering
\lstinputlisting[language=HTML, style=customhtml]{moduleSchema2.html}
\caption{Module Schema enriched with annotations and user text}
\label{figure:moduleTemplate2}
\end{figure}

\begin{figure}[h]
\centering
\lstinputlisting[language=HTML, style=customhtml]{viewSchema2.html}
\caption{Module Viewtype view Schema enriched with annotations and user text}
\label{figure:viewTemplate2}
\end{figure}

