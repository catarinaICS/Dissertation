%!TEX root = ../dissertation.tex

\chapter{Architecture Analysis}
\label{chapter:architecture}
Chapter \ref{chapter:domainModel} introduced the domain model of the Software Architectures concepts present in the developed solution, and the templates in which they are included.

In this chapter, it is shown how these templates are filled with information extracted from a software description article.

The next sections will present two entities from the system: the Document and the Annotation. These entities have a major role in the identification of Software Architectures concepts from the domain model, and the enrichment of the templates for those concepts. 

\section{Document}
\label{section:document}

The Document entity in this system corresponds to an article which describes a software system and is read and analysed in the practical classes of the Software Architectures course. This entity saves the article text and its source, so it can be parsed locally in the application

Figure \ref{figure:documentEntity} shows the Document entity in the system. The ``title'' attribute corresponds to the title of the article, the ``url'' saves the article's original URL, and the ``content'' saves the article's contents.

\begin{figure}[h]
\centering
\renewcommand {\umltextcolor}{black}
\renewcommand {\umlfillcolor}{none}
\renewcommand {\umldrawcolor}{black}

\begin{tikzpicture}
\begin{class}[text width=3cm ]{Document}{0,0}
	\attribute { title : String }
	\attribute { url : String }
	\attribute { content : String }
\end{class}
\end{tikzpicture}
\caption{The Document Entity}
\label{figure:documentEntity}
\end{figure}

As the main purpose of this application is to create a structured representation of the document using the templates described in Section \ref{section:templates}, the instances of the main concepts, from which the templates will be generated, must be associated to the Document, as shown in Figure \ref{figure:documentConcept}. A concept describes a part of one and only one document, hence the '1' cardinality.  

\begin{figure}[h]
\centering
\renewcommand {\umltextcolor}{black}
\renewcommand {\umlfillcolor}{none}
\renewcommand {\umldrawcolor}{black}

\begin{tikzpicture}
\begin{class}[text width=3cm ]{Concept}{3,-4}
	\attribute{}
\end{class}

\begin{class}[text width=3cm ]{Document}{0,0}
	\attribute { title : String }
	\attribute { url : String }
	\attribute { content : String }
\end{class}

\association{Document}{}{1}{Concept}{}{*}
\end{tikzpicture}
\caption{Association between Document and Concept entities}
\label{figure:documentConcept}
\end{figure}

\section{Annotation}
\label{section:annotation}

An annotation consists of a portion of text selected from the article, enriched with a tag and other information. This selected text can correspond partially or totally to the specification of a Software Architecture concept. For example, the description of the Source of Stimulus of a Scenario may be found in a single paragraph of the article, or in a set of single sentences scattered along the whole article. 

Figure \ref{figure:annotationEntity} shows the Annotation entity in the system. The ``annotation'' field saves a Json representation of the annotation data such as the quote from the text and the given tag. The ``tag'' field stores the tag given to the annotation. This tag corresponds to a Software Architectures concept, such as ``Stimulus'' or ``Module''.

\begin{figure}[h]
\centering
\renewcommand {\umltextcolor}{black}
\renewcommand {\umlfillcolor}{none}
\renewcommand {\umldrawcolor}{black}

\begin{tikzpicture}
\begin{class}[text width=4cm ]{Annotation}{0,0}
	\attribute { annotation : String }
	\attribute { tag : String }
\end{class}
\end{tikzpicture}
\caption{The Annotation Entity}
\label{figure:annotationEntity}
\end{figure}

Annotations are added to the system by selecting a portion of text in the parsed document and setting it as an annotation. Therefore, an annotation belongs to one, and only one document. Figure \ref{figure:documentAnnotation} shows the association between the Annotation and Document entities which reflects the mentioned constraints.

\begin{figure}[h]
\centering
\renewcommand {\umltextcolor}{black}
\renewcommand {\umlfillcolor}{none}
\renewcommand {\umldrawcolor}{black}

\begin{tikzpicture}
\begin{class}[text width=4cm ]{Annotation}{2,-5}
	\attribute { annotation : String }
	\attribute { tag : String }
\end{class}

\begin{class}[text width=3cm ]{Document}{0,0}
	\attribute { title : String }
	\attribute { url : String }
	\attribute { content : String }
\end{class}

\association{Document}{}{1}{Annotation}{}{*}
\end{tikzpicture}
\caption{Association between Document and Annotation entities}
\label{figure:documentAnnotation}
\end{figure}

Annotations are used to enrich the templates described in Section \ref{section:templates}. Associating the annotations with the respective instances of the concepts allows for that enrichment. All the concepts, not only the main ones, can have associated annotations.

As mentioned in the previous section, an annotation can partially or fully describe a concept, therefore a concept can be associated with multiple annotations. 

Figure \ref{figure:annotationConcept} shows the association between the Source of Stimulus and the Annotation entities as an example, as all the concepts in the domain model of Figure \ref{figure:abstractDomainModel} have a similar association.

\begin{figure}[h]
\centering
\renewcommand {\umltextcolor}{black}
\renewcommand {\umlfillcolor}{none}
\renewcommand {\umldrawcolor}{black}

\begin{tikzpicture}
\begin{class}[text width=4cm ]{Annotation}{0,0}
	\attribute { annotation : String }
	\attribute { tag : String }
\end{class}

\begin{class}[text width=3cm ]{Source Of Stimulus}{4,-4}
	\attribute{}
\end{class}

\association{Source Of Stimulus}{}{0..1}{Annotation}{}{*}
\end{tikzpicture}
\caption{Association between Source Of Stimulus and Annotation entities}
\label{figure:annotationConcept}
\end{figure}

\section{Interface Flow}
\label{section:interfaceFlow}

The associations between the entities described in the previous sections are created after a set of interactions with the system. These interactions are:

\begin{enumerate}
\item After navigation to the document page, a portion of text is selected. The AnnotatorJS interface is prompted, and the user selects a tag to describe the selected text;

\item Upon submission, an instance of ``Annotation'' is created and associated with the instance of the document;

\item By mouse hovering over the highlighted text, another AnnotatorJS interface is prompted, and the user can associate this annotation to a domain entity by clicking ``Add this annotation to the structured representation'';

\item Upon clicking the link, a modal window is shown and, according to the tag given to the annotation, the user can select either an existant Scenario, View or Module to add this annotation to, or create a new instance. 

Initially there are no created instances, and upon creation the instance is associated with the document. When a new Scenario is created, their elements are instantiated as well;

\item After selection, the annotation is associated with the selected element. In the case of Scenarios, if the annotation describes any of the scenario elements, it will be associated with the instance of the corresponding element. 

For example, an annotation tagged as ``Stimulus'' will be associated with the instance of the Stimulus entity that is associated with the selected Scenario.

\item The user is then redirected to the template of the selected element. Inside the template, the user can delete the annotation, move it to other element, delete the element itself and add text descriptions to the template. 

In the Scenario template the user can also add Tactics, in the Module template the user can add relations with other Modules, and in the View template the user can add elements to the view.
\end{enumerate}

Figures \ref{figure:scenarioTemplate2}, \ref{figure:moduleTemplate2} and \ref{figure:viewTemplate2} show the Schemas presented in Section \ref{section:templates} of Chapter \ref{chapter:domainModel}, now including the annotations and the user text.

\begin{figure}[h]
\centering
\lstinputlisting[language=HTML, style=customhtml]{scenarioSchema2.html}
\caption{Scenario Schema enriched with annotations and user text}
\label{figure:scenarioTemplate2}
\end{figure}

\begin{figure}[h]
\centering
\lstinputlisting[language=HTML, style=customhtml]{moduleSchema2.html}
\caption{Module Schema enriched with annotations and user text}
\label{figure:moduleTemplate2}
\end{figure}

\begin{figure}[h]
\centering
\lstinputlisting[language=HTML, style=customhtml]{viewSchema2.html}
\caption{Module Viewtype view Schema enriched with annotations and user text}
\label{figure:viewTemplate2}
\end{figure}















