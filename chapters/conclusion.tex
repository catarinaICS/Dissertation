%!TEX root = ../dissertation.tex

\chapter{Conclusion}
\label{chapter:conclusion}

The success of social software is very dependent on the people that use it, since it is focused on the interactions and collaboration between their users.

Developing a platform including not only a way to structure knowledge but also a set of social features is a very ambitious task, and the focus of this thesis was the knowledge structuring part, which is very important for the context of the Software Architectures course.

Although the main feature of the developed solution is the possibility of extracting structured information from an unstructured source, this is done by a set of users authenticated within the system, which can collaboratively read and work over the same text.

Despite its simplicity and the lack of volunteers to test it, overall the platform proved that it provides a way for students to work together. However, there are still many features that could be added to make it more collaborative and more social.





