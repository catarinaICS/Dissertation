%!TEX root = ../dissertation.tex

\chapter{Implementation}
\label{chapter:implementation}

The application developed follows a Model-View-Controller architecture.

This architecture is composed of three elements \cite{krasner1988description}:
\begin{itemize}
\item The Model, which captures and stores information about the application domain;
\item The View, which generates output based on the model state;
\item The Controller, which provides interaction between the Views and the Model. It is able to modify the Model and update the corresponding View.
\end{itemize}

The standard interaction cycle in this architecture is the following \cite{krasner1988description,reenskaug2009dci}:
\begin{itemize}
\item A user interacting with the application sends a request to the Controller.

\item The Controller communicates with the Model to apply the desired changes.

\item The Model is updated.

\item The Controller notifies the corresponding View so it can be updated with the new version of the Model.
\end{itemize}

The next sections will describe the solution implementation based on these three elements.


\section{Model}
\label{section:model}
The Model of the implemented system is based on the Domain Model presented in chapter \ref{chapter:domainModel}, but a few modifications were made to facilitate programming.

\subsection{Annotations}
\label{subsection:modelAnnotations}
As seen in Chapter \ref{chapter:architecture}, the Document is one of the most important entities, because it aggregates all the annotations created over the text and all the Software Architectures concepts elicited.

Similarly to what is shown in figure \ref{figure:documentAnnotation} of chapter \ref{chapter:architecture}, in the implemented model, the Document entity is connected to the Annotation. However, an annotation is created by an user authenticated in the system, and this information is also present in the domain, as shown in figure \ref{figure:modelDocUserAnnot}.

\begin{figure}
\centering
\renewcommand {\umltextcolor}{black}
\renewcommand {\umlfillcolor}{none}
\renewcommand {\umldrawcolor}{black}

\begin{tikzpicture}
\begin{class}[text width=3cm ]{Document}{0,0}
	\attribute { title : String }
	\attribute { url : String }
	\attribute { content : String }
\end{class}

\begin{class}[text width=3.5cm ]{Annotation}{-2,-5}
	\attribute { annotation : String }
	\attribute { tag : String }
\end{class}

\begin{class}[text width=3.5cm ]{User}{-6,0}
	\attribute { username : String }
	\attribute { password : String }
	\attribute { firstName : String }
	\attribute { lastName : String}
	\attribute { type : String}
\end{class}
\association {Document}{document}{1}{Annotation}{annotations}{0..*}
\association {User}{owner}{1}{Annotation}{}{0..*}

\end{tikzpicture}
\caption{Document and Annotations in the implemented Model}
\label{figure:modelDocUserAnnot}
\end{figure} 
 
\subsection{Scenarios}
\label{subsection:modelScenarios}
Figure \ref{figure:modelScenarios} shows how scenarios are represented in the implemented model. This diagram is similar to what is shown in figure \ref{figure:abstractDomainModel}, with a small difference regarding the Scenario Quality Requirement and Tactics. 

The diagram in section \ref{section:model} shows an association between Tactic and Quality Requirement, in the sense that there are a set of tactics used to achieve that quality requirement. 

Regarding the model implemented, a quality requirement instance is associated with the scenario upon its creation, but tactics, which are added to the scenario afterwards, are only associated to that scenario and nothing else. The Quality Requirement and the Tactics are only related in the sense that the quality requirement associated upon scenario creation will define which tactics can be added to the scenario. This will be described further.

The Scenario is a main concept of the domain, and therefore is connected to the Document. The ``identifier'' parameter in the entities stores the name of the concept that corresponds to that entity, and often corresponds to the name of the entity itself. For example, the identifier in all instances of Scenario is ``Scenario'' and the identifier in all instances of ``SrcOfStimulus'' is ``Source Of Stimulus''. This was done to facilitate programming. 

\begin{figure}[h]
\centering
\renewcommand {\umltextcolor}{black}
\renewcommand {\umlfillcolor}{none}
\renewcommand {\umldrawcolor}{black}

\begin{tikzpicture}
\begin{class}[text width=3cm ]{Document}{9,-2.5}
	\attribute { title : String }
	\attribute { url : String }
	\attribute { content : String }
\end{class}

\begin{class}[text width=3cm ]{Scenario}{9,-7.5}
	\attribute { name : String}
	\attribute { identifier : String }
	\attribute { text : String }
\end{class}

\begin{class}[text width=3cm ]{QualityRequirement}{9,-13.5}
	\attribute { name : String}
\end{class}

\begin{class}[text width=3cm ]{SrcOfStimulus}{0,-1.5}
	\attribute { identifier : String }
	\attribute { text : String }
\end{class}	

\begin{class}[text width=3cm ]{Stimulus}{0,-3.5}
	\attribute { identifier : String }
	\attribute { text : String }
\end{class}

\begin{class}[text width=3cm ]{Artifact}{0,-5.5}
	\attribute { identifier : String }
	\attribute { text : String }
\end{class}

\begin{class}[text width=3cm ]{Environment}{0,-7.5}
	\attribute { identifier : String }
	\attribute { text : String }
\end{class}

\begin{class}[text width=3cm ]{Response}{0,-9.5}
	\attribute { identifier : String }
	\attribute { text : String }
\end{class}

\begin{class}[text width=3cm ]{ResponseMeasure}{0,-11.5}
	\attribute { identifier : String }
	\attribute { text : String }
\end{class}

\begin{class}[text width=3cm ]{Tactic}{0,-13.5}
	\attribute { identifier : String }
	\attribute { text : String }
\end{class}

\association {Scenario}{}{1}{SrcOfStimulus}{}{0..1}
\association {Scenario}{}{1}{Stimulus}{}{0..1}
\association {Scenario}{}{1}{Artifact}{}{0..1}
\association {Scenario}{}{1}{Environment}{}{0..1}
\association {Scenario}{}{1}{Response}{}{0..1}
\association {Scenario}{}{1}{ResponseMeasure}{}{0..1}
\association {Scenario}{}{1}{Tactic}{}{*}
\association {Scenario}{}{*}{Document}{}{0..1}
\association {Scenario}{}{*}{QualityRequirement}{}{0..1}

\end{tikzpicture}
\caption{Scenario Representation in the implemented Model}
\label{figure:modelScenarios}
\end{figure}

To facilitate programming, all Scenario elements are a subclass of a ``ScenarioElement'' class, as seen in the example in Figure \ref{figure:modelScenarioElement}. This class contains a series of methods that are common to all the Scenario Elements. A few of those methods are shown in the figure. However, this class is not associated with the Scenario entity. Its purpose is only to facilitate programming by aggregating common methods. 

\begin{figure}
\centering
\renewcommand {\umltextcolor}{black}
\renewcommand {\umlfillcolor}{none}
\renewcommand {\umldrawcolor}{black}

\begin{tikzpicture}

\begin{class}[text width=3cm ]{ScenarioElement}{0,0}
	\attribute { identifier : String }
	\attribute { text : String }
	\operation { getIdentifier() : String }
	\operation { getText() : String}
\end{class}

\begin{class}[text width=3cm ]{SrcOfStimulus}{-2.1,-3}
	\inherit{ScenarioElement}
	\attribute { identifier : String }
	\attribute { text : String }
\end{class}

\draw [umlcd style] (0,-3.7) node {...};

\begin{class}[text width=3cm ]{ResponseMeasure}{2.1,-3}
	\inherit{ScenarioElement}
	\attribute { identifier : String }
	\attribute { text : String }
\end{class}		
	
\end{tikzpicture}
\caption{The ScenarioElement entity}
\label{figure:modelScenarioElement}
\end{figure}

As mentioned in Section \ref{section:annotation}, all domain entities have associated annotations. Figure \ref{figure:modelScenariosAnnotations} shows how scenarios and their elements are related with the annotations in the implemented model. This diagram is similar to the one in figure \ref{figure:annotationConcept} and shows the association between the Scenario and ScenarioElement with the Annotation.

\begin{figure}
\centering
\renewcommand {\umltextcolor}{black}
\renewcommand {\umlfillcolor}{none}
\renewcommand {\umldrawcolor}{black}

\begin{tikzpicture}
\begin{class}[text width=3.5cm ]{Annotation}{0,0}
	\attribute { annotation : String }
	\attribute { tag : String }
\end{class}
\begin{class}[text width=3cm ]{ScenarioElement}{-2,-4}
	\attribute { identifier : String }
	\attribute { text : String }
\end{class}

\begin{class}[text width=3cm ]{Scenario}{2,-4}
	\attribute { name : String }
	\attribute { identifier : String }
	\attribute { text : String }
\end{class}	
\association {Scenario}{}{0..1}{Annotation}{}{*}
\association {ScenarioElement}{}{0..1}{Annotation}{}{*}
\end{tikzpicture}
\caption{Scenarios and Scenario Elements and their relation with Annotations in the implemented model}
\label{figure:modelScenariosAnnotations}
\end{figure}

\subsection{Views}


Figure \ref{figure:domainModel5} shows how Views and Modules relate with the Document. A View belongs, as mentioned in Chapter \ref{chapter:domainModel}, to a certain viewtype, which is refined by a specific style. These details are present in the entity attributes. Views are associated with the corresponding Document after creation.
A View from the Module Viewtype contains Modules and the relationships between them. 
These Modules are created in a similar way as Scenarios or Views, and associated with the Document. Afterwards, they can be associated with Views.

\begin{figure}
\centering
\renewcommand {\umltextcolor}{black}
\renewcommand {\umlfillcolor}{none}
\renewcommand {\umldrawcolor}{black}
\begin{tikzpicture}
\begin{class}[text width=3.5cm ]{View}{0,0}
	\attribute { name : String }
	\attribute { viewtype : String }
	\attribute { style : String }
	\attribute { text : String }
\end{class}
\begin{class}[text width=3cm ]{Module}{-3,-4}
	\attribute { name : String }
	\attribute { identifier : String }
	\attribute { text : String }
\end{class}

\begin{class}[text width=3cm ]{Document}{3,-4}
	\attribute { title : String }
	\attribute { url : String }
	\attribute { content : String }
\end{class}	
\association {View}{}{*}{Document}{}{0..1}
\association {Module}{}{*}{Document}{}{0..1}
\association {Module}{}{*}{View}{}{*}

\end{tikzpicture}

\caption{View and Module in the implemented Model}
\label{figure:domainModel5}
\end{figure}

Similarly to other entities, both the Module and View can have associated annotations, as seen in Figure \ref{figure:domainModel6}.

\begin{figure}
\centering
\renewcommand {\umltextcolor}{black}
\renewcommand {\umlfillcolor}{none}
\renewcommand {\umldrawcolor}{black}
\begin{tikzpicture}
\begin{class}[text width=3.5cm ]{View}{0,0}
	\attribute { name : String }
	\attribute { viewtype : String }
	\attribute { style : String }
	\attribute { text : String }
\end{class}
\begin{class}[text width=3cm ]{Module}{-3,-4}
	\attribute { name : String }
	\attribute { identifier : String }
	\attribute { text : String }
\end{class}

\begin{class}[text width=3cm ]{Annotation}{3,-4}
	\attribute { annotation : String }
	\attribute { tag : String }

\end{class}	
\association {View}{}{0..1}{Annotation}{}{*}
\association {Module}{}{0..1}{Annotation}{}{*}
\association {Module}{}{*}{View}{}{*}

\end{tikzpicture}

\caption{Domain Model: Views and Modules relationship with Annotations}
\label{figure:domainModel6}
\end{figure}

\begin{figure}
\centering
\renewcommand {\umltextcolor}{black}
\renewcommand {\umlfillcolor}{none}
\renewcommand {\umldrawcolor}{black}
\begin{tikzpicture}

\begin{class}[text width=3cm ]{Module}{-3,-4}
	\attribute { name : String }
	\attribute { identifier : String }
	\attribute { text : String }
\end{class}
	
%\association {Module}{}{0..1}{Module}{}{*}
\draw [umlcd style] (Module) -- (-3,-7) -- (0,-7) -- (0,-4.95) -- (Module);



\end{tikzpicture}

\caption{Domain Model: Module relationships (INCOMPLETO)}
\label{figure:domainModel6}
\end{figure}

\section{Controllers}

\subsection{Annotation Controller}
The Annotation Controller provides a way to manage the annotations added to a document. It communicates with the Domain Model to create, remove or update Annotations. The particularity of this controller is that it provides a REST API, which is used to retrieve a JSON representation of the annotations stored in the model and display it along with the document text. 
The endpoints provided by this controller are described in Table \ref{table:endpoints}.
\begin{table}
\begin{tabular}{ | l | l | l |p{7.7cm}|}
    \hline
    Name & Method & Endpoint & Description \\ \hline
    Index & GET & ../store/annotations & \parbox[t]{8cm}{Returns the set of annotations associated with a \\specific document }\\ \hline
    Read & GET & ../store/annotations/id & Returns the annotation with the specific id \\ \hline
    Create & POST & ../store/annotations & \parbox[t]{8cm}{Creates a new annotation, stores it in the model, \\and redirects to the Read endpoint} \\ \hline
    Update & PUT & ../store/annotations/id & \parbox[t]{8cm}{Updates the annotation with the given id and \\redirects to the Read endpoint} \\ \hline
    Delete & DELETE & ../store/annotations/id & \parbox[t]{8cm}{Removes the annotation with the given id. The \\response is a HTTP/1.0 204 NO CONTENT.} \\ \hline
  \end{tabular}
  \caption{REST API provided by the Annotation Controller}
  \label{table:endpoints}
\end{table}


\subsection{Document Controller}
The Document Controller handles the requests to view, add or remove a document from the system.

\subsection{Scenario Controller}
The Scenario Controller handles the construction of templates for Scenarios. All requests to create a new scenario, link document annotations to a specific scenario or delete a scenario are handled in this controller.
