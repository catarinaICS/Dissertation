%!TEX root = ../dissertation.tex

\chapter{Introduction}
\label{chapter:introduction}

Analyzing and discussing big, real, open-source and highly complex software system is a very important part of the Software Architectures course at Instituto Superior T\'{e}cnico. In the course context, students must apply concepts and techniques for design and analysis of software architectures to descriptions of real systems. 

However, applying these concepts and techniques is not a very easy task, and often students have questions and doubts regarding these descriptions. These questions sometimes require not only consulting the course bibliography, but also discussing with peers or asking questions to teachers. This thesis focuses on providing a solution for this problem with the use of social software and knowledge structuring strategies.

This document is organized as follows: Chapter \ref{chapter:problemdescription} gives a more detailed description about the problem of applying theoretical concepts to practical examples in the context of the Software Architectures course. Chapter \ref{chapter:objectives} elicits the main goals of this thesis. Chapter \ref{chapter:relatedwork} presents the state-of-the-art in the areas of Social Software and Knowledge Structuring. Chapter \ref{chapter:solution} gives a small description of the developed solution. Chapters \ref{chapter:domainModel}, \ref{chapter:architecture} and \ref{chapter:implementation} describe the solution domain model, architectural details and implementation. Finally, Chapter \ref{chapter:evaluation} will show the assessment of the developed solution and \ref{chapter:futureWork} will describe what can still be added to it.