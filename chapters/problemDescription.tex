%TEX root = ../dissertation.tex

\chapter{Problem Description}
\label{chapter:problemdescription}
The course of Software Architectures teaches students the most important concepts in the field of software architectures and applies these concepts to real-life software systems.

The practical component of this course (where the theory is applied) is done by analyzing documents/articles that describe the architectures of real-life systems and applying the concepts learned in theory lessons: extracting stakeholders, scenarios, tactics, views, etc.
This analysis is either done by students with the help of the teacher, during practical classes, or in work groups of usually three students, that read, discuss and analyze software descriptions together and present their results to both the teacher and the rest of the class.

Understanding the analysis done and how it was done is very important, since it means students can apply the concepts learned not only in the written exam to pass the course, but also in the future, in other real-life software systems.
However, there are several issues regarding the application of theory concepts in this course:
\begin{itemize}
\item Each year, there are around one-hundred students signed in the course. All these students have roughly the same Computer Science background knowledge. However, not all students have the same maturity: while some may quickly understand what is taught in theory lessons, others may need some more time to assimilate what was taught.

\item The output of the analysis done to software descriptions (the scenarios, etc. extracted) is not available in a consistent way: the analysis done in class is available for students if they took their own notes, and the analysis done by groups is available if:
\begin{enumerate}
\item Students took notes of their colleagues' presentations 

\item Groups share their presentation slides among them. Slides may include errors pointed out by the teacher, but not corrected.

\end{enumerate} 
\item The case description is usually a fairly long document (from ten to twenty pages approximately). The task of carefully reading and understanding all the text and do a mapping between the concrete descriptions in the text and the abstract concepts learned in theory classes is not easy, as the parts of the text that map to concepts are not always evident.
\item The architectural elements extracted from a single software description are usually scattered along the whole text, and it is not evident the connection between all the elements.
\end{itemize}