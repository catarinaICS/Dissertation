\chapter{Software Architectures Concepts}
\label{chapter:concepts}
Before describing the architecture of the developed solution, it is necessary to provide a small introduction to the most important concepts from the Software Architectures course.
\section{Scenarios}

A scenario is used to capture and express quality requirements of a system. It consists of six parts \cite{bass2003software}:
\begin{itemize}
\item \textbf{Source of Stimulus:} Some entity (a human, a computer or any other actuator) that generates the stimulus;
\item \textbf{Stimulus:} A condition that arrives at the system;
\item \textbf{Environment:} The system condition when the stimulus occurs;
\item \textbf{Artifact:} Part of the system that was stimulated;
\item \textbf{Response:} Activity undertaken after the arrival of the stimulus;
\item \textbf{Response Measure:} when the response occurs, it should be measurable in some way, so the requirement can be tested;
\end{itemize}

Tactics are design decisions used to achieve the quality requirements expressed by the scenarios.

\section{Views}
A view is a representation of a set of system elements and the relationships associated with them \cite{clements2003documenting}. This set of elements and relationships is constrained by viewtypes.

A Viewtype defines the element types and relationship types used to describe the architecture of a software system from a particular perspective\cite{clements2003documenting}. Viewtypes refine into styles.

An architectural style is a specialization of element and relation types, together with a set of constraints on how they can be used\cite{clements2003documenting}.

Views can fall into three viewtype categories\cite{clements2003documenting}:
\begin{itemize}
\item \textbf{Module Viewtype:} document the system principal units of implementation. 

The elements of this viewtype are the \textit{Modules}, which is are implementation units. 

Relationships between modules can be of type \textit{``Is-part-of''}, which defines a part-whole relationship, \textit{``Depends on''}, which defines dependency relations, and \textit{``Is-a''}, which defines a generalization/specialization relationship.

\item \textbf{Component \& Connector Viewtype:} document the system units of execution. 

The elements are the Components, which are the principal processing units and data stores, and the Connectors, which are pathways of interaction between components. 

The relationships can be of type \textit{``Attachment''}, which associate components to connectors, and \textit{``Interface''} Delegation, which associates component ports to other ports from an ``internal architecture''- and similarly for the connector.

\item \textbf{Allocation Viewtype:} document the relationships between a system's software and its development and execution environments. 

The elements are the \textit{Software Element} and the \textit{Environmental Element}. 

The relationships are of type \textit{``Allocated-to''}, which means that a software element is allocated to an environmental element. 
\end{itemize}



