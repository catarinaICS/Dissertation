%!TEX root = ../dissertation.tex

\chapter{Future Work}
\label{chapter:futureWork}

The analysis of the state-of-the-art on Social Software and Knowledge Structuring done in Chapter \ref{chapter:relatedwork} elicits a set of features that should be present in the solution.

However, developing a platform that provides structuring knowledge from an unstructured source and all the described social features was a very ambitious goal for the time span available. 

The most important feature for this platform was, in fact, the knowledge structuring part, as the main goal of this thesis is to provide a solution to facilitate the analysis of the software description article and help the students to find the correlation between the concepts learned in the theoretical classes and these articles analyzed in the practical ones.

Therefore, the developed solution focused on the association of tags from a closed vocabulary to parts of text, and the creation of a semi-structured representation of the software description articles, where the elements of the Software Architectures are clearly distinguished and described by the corresponding parts of text and user text.

Finding the correct way to make these associations and representations was not a easy task, and there are still details of the developed solution that could be improved and new features that could be added in the future.

\section{Allocation Viewtype}

The Module and Component \& Connector Viewtypes are the most common views present in a software description article, and were the ones that got most attention during development.

Allocation viewtype elements should be added to the platform in a future version, so Allocation Viewtype Views can be added to the structured representation of the article.

\section{Graphical Representation of Views}

The developed solution allows for the creation and representation of Modules, Components, Connectors and the Views that include these elements. 

Module and Component \& Connector Viewtypes are only represented textually in the platform. However, a good way to represent these views so that the relations between the elements are visible is to represent them in graphical diagrams. 

Features to create or upload graphical representations of Views would be very useful for the platform as a way to improve comprehension about the Views and their elements.

\section{Social Elements}

Sections \ref{section:relatedWorkHoneycomb}, \ref{section:relatedWorkPersuasive} and  \ref{section:relatedWorkRepSys} of Chapter \ref{chapter:relatedwork} describing the Related Work list a series of social features that could be useful in the platform.

The developed platform contains social elements and allows for collaboration, as students are individually identified within the system and work together over a shared article. However, there are several other social software elements that could enrich the platform as a social system:

\begin{itemize}
\item \textbf{Groups:} This is the most important social aspect that could be added to the platform, as besides reading and analyzing articles in the practical classes, students must also read and analyze other articles for Group Assignments. The existence of features for the registration of work groups and having articles with group-access only could make it easier for the students to work on their group assignments.

\item \textbf{Q\&A and/or Discussion Forums:} It may not always be clear for a student which parts of the text correspond to which concepts. Although they can simply ask the teacher or a colleague, it could also be very useful to have a central place to ask questions and discuss decisions. Having a discussion forum and/or a Q\&A system could also improve collaboration between students.

\item \textbf{Reputation System: } Although its role would only be motivational, it is important that students feel motivated to participate in the platform, and a reputation system, assigning scores to, for example, annotations or comments, could be useful in this platform.

\item \textbf{Spotlight: } Having a spotlight of the students with most participations and/or the highest reputation scores would provide not only extra motivation, but also promote some healthy competition between students.
\end{itemize}
