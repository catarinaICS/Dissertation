%!TEX root = ../dissertation.tex

\chapter{Appendix B - Survey Results}
\label{appendix:appendix_b}

\textbf{Regarding the application usability, how would you rate the following:}
\begin{table}[h]
\centering
\scriptsize
\begin{tabular}{|c|c|c|c|c|c|c|}
    \hline
     \textbf{Participants} & \parbox[c][1cm]{1.5cm}{\textbf{Creating} \\ \textbf{Annotations}} & \parbox[c][1cm]{1.5cm}{\textbf{Creating}\\ \textbf{Scenarios}} & \parbox[c][1cm]{3cm}{\textbf{Associating Annotations} \\ \textbf{with Scenarios}} & \parbox[c][1cm]{2cm}{\textbf{How a Scenario} \\ \textbf{is represented}} & \parbox[c][1cm]{1.5cm}{\textbf{Editing}\\ \textbf{Scenarios}} & \parbox[c][1cm]{2cm}{\textbf{Structured} \\ \textbf{Representation}}\\ \hline
     Participant \#1 & 7 &	9 &	8 & 5 & 4 & 6\\ \hline
	Participant \#2 & 4 & 9 & 1 0 & 7 & 6 & 5\\ \hline
	Participant \#3 & 10 & 10 & 10 & 9 & 9 & 9\\ \hline
	Participant \#4 & 10 & 8 & 8 & 7 & 9 & 7\\ \hline
	Participant \#5 & 10 & 8 & 7 & 8 & 8 & 6\\ \hline
\end{tabular}
\end{table}

\textbf{Application Usefulness:}	
\begin{table}[h]
\centering
\scriptsize
\begin{tabular}{|c|c|c|c|}
    \hline
     \textbf{Participants} & \parbox[c][2cm]{5cm}{\textbf{How much does the application facilitate the application of the concepts you learned in the theoretical classes to the concrete examples (such as Graphite)?}} & \parbox[c][2cm]{3cm}{\textbf{How useful do you think the application could be for your group assignments?}} & \parbox[c][2cm]{3cm}{\textbf{How useful do you think this application would be in the context of Practical Classes?}} \\ \hline
     Participant \#1 & 9 & 9 & 9\\ \hline
Participant \#2 & 8 & 10 & 10\\ \hline
Participant \#3 & 10 & 9 & 10\\ \hline
Participant \#4 & 7 & 7 & 7\\ \hline
Participant \#5 & 7 & 9 & 9\\ \hline
\end{tabular}
\end{table}

\begin{table}[h]
\begin{flushleft}
\textbf{Opinions:} \\
\end{flushleft}

\centering
\scriptsize
\begin{tabular}{|c|c|c|c|}
    \hline
     \textbf{Participants} & \parbox[c][2cm]{2cm}{\textbf{What did you like the most about the application?}} & \parbox[c][2cm]{4cm}{\textbf{What improvements do you think that can be done to the application?}} & \parbox[c][2cm]{4cm}{\textbf{Other feedback}} \\ \hline
Participant \#1 & \parbox[c][2cm]{2cm}{It provides a quick and convenient way of constructing scenarios}& \parbox[c][2cm]{4cm}{There should be a way to "view" the scenario itself, with the diagram that is showed in the lectures. Also, the system needs to be more intuitive, particularly when it comes to selecting sentences for the scenarios} &\parbox[c][2cm]{4cm}{ Would it be possible to view the most common options for each part (stimulus/response/etc.) once you've selected what type of scenario it is (availability/performance/etc.)?} \\ \hline
Participant \#2 & \parbox[c][5cm]{2cm}{I liked most how easy it was to create and edit scenarios} & \parbox[c][2cm]{4cm}{I had trouble when trying annotate a sequence of letters when that sequence overlapped with another annotation. For example, I added a long sequence for the stimulus but then I wanted to use a subsequence of that sequence in the scenario description. I dont remember correctly, but I think I also couldn't add more than one entry per type (like stimulus). And also I couldnt add custom hand-made entries? Sometimes the there is no explicit ""response"", but we can write it and justify it with some part of the text"}  & \parbox[c][2cm]{4cm}{I think this kind of application can be used in other domains. For example it could be used to create character profiles from a narrative book, etc...} \\ \hline
Participant \#3 & \parbox[c][2cm]{2cm}{It is possible for different people to analyze the same document, and its easy to use.} & \parbox[c][2cm]{4cm}{Support sharing files with different groups.} & N/A \\ \hline
Participant \#4 & \parbox[c][2cm]{2cm}{The annotation method} & \parbox[c][2cm]{4cm}{Make multiple copies of the article to bee only seen by each group so that one group doesn't get confused or get influenced by others.} & \parbox[c][2cm]{4cm}{Good luck with the rest of the project!} \\ \hline
Participant \#5 & \parbox[c][2cm]{2cm}{It is extremely easy and intuitive to make annotations, just like on paper.} & \parbox[c][6cm]{4cm}{When adding a tactic, only the tactics presented in the book 'Software Architecture in Practice' are available. I think it could be useful if the user could add new tactics instead of just selecting one of those. The menu bar could be improved by placing there buttons for the main application functionalities. For example, the 'View Structured Representation' link that appears just below the menu bar could be placed on the menu bar instead.An useful functionality could be generating a graphical representation of the scenarios, like those ones on the book with the artifact and the arrows. Another useful (yet difficult to implement, perhaps) functionality would be the possibility to upload new case studies to use in the application.} & Nice work :) \\  \hline		
\end{tabular}
\end{table}

